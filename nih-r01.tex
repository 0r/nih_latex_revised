\documentclass[12pt]{article}
\usepackage[sort&compress]{natbib}
\usepackage[yyyymmdd,hhmmss]{datetime}
\usepackage[hidelinks]{hyperref}

% Use Helvetica font.
\usepackage{helvet}
\renewcommand{\familydefault}{\sfdefault}

% Use 1/2-inch margins.
\usepackage[margin=0.5in]{geometry}

% No page numbers.
\pagestyle{empty}

% Allow wrapping of text around figures.
\usepackage{wrapfig}
\usepackage{url}

% A space-saving version of \paragraph.
\newcommand{\myparagraph}[1]{\vspace{0.5ex}\noindent{\bf #1} ~~} 

% Make captions have small font.
\usepackage[font=scriptsize]{caption}

\newcommand{\instructions}[1]{}
% Before submitting, comment out this line.
\renewcommand{\instructions}[1]{{\scriptsize \sc #1}}

\usepackage{xcolor}
\newcommand{\fixme}[1]{}
\renewcommand{\fixme}[1]{{\color{red} {#1}}\color{black}}

% Rotate 45 degrees.
\usepackage{rotating}
\newcommand{\rdg}[1]{\begin{rotate}{45} #1 \end{rotate}}

% Display a checkmark.
\usepackage{amssymb}
\newcommand{\vv}{\checkmark}

% Compact titles.
\usepackage[compact]{titlesec}

% Compacted lists.  Use itemize* or enumerate* for compact version.
\usepackage{mdwlist}

% Reduce space after figures.
\setlength{\belowcaptionskip}{-10pt}

\usepackage{authblk} % Better formatting of author list.

% The title has an 81-character maximum.
\title{INSERT TITLE HERE}

\author[1]{My name}

\affil[1]{My affiliation}

\date{}

\begin{document}

\maketitle

\begin{center}
An NIH R01 proposal, submitted in response to \\
{\em \href{https://grants.nih.gov/grants/guide/pa-files/PA-20-185.html}{PA-20-185}: NIH Research Project Grant}.
\end{center}

\vfill

\noindent
Deadline for draft science and final budget: \\
Deadline for submission to OSP: \\
Deadline for submission to NIH: \\

{\scriptsize Compiled on \today\ at \currenttime.}
%%%%%%%%%%%%%%%%%%%%%%%%%%%%%%%%%%%%%%%%%%%%%%%%%%%%%%%%%%%%%%%%%%%%%%%%%%%%%

\maketitle

%%%%%%%%%%%%%%%%%%%%%%%%%%%%%%%%%%%%%%%%%%%%%%%%%%%%%%%%%%%%%%%%%%%%%%%%%%%%%%%
\clearpage
\section*{PHS Assignment Request Form}

\subsection*{Awarding component assignment suggestions}

\instructions{If you have a suggestion for an awarding component (e.g., NIH Institute/Center) assignment, use the link below to identify the appropriate short abbreviation (e.g., ``NCI'' for National
  Cancer Institute) and enter it below in the boxes for ``Suggested Awarding Components''. All suggestions will be considered; however, not all assignment suggestions can be honored.

  Information about Awarding Component can be found here: \url{https://grants.nih.gov/grants/phs_assignment_information.htm\#AwardingComponents}}

\subsection*{Study section assignment suggestions}

\instructions{If you have a suggestion for a study section assignment, use the link below to identify a study section(s). Enter the short abbreviation for that study section in the boxes for ``Suggested
Study Sections.'' Remove all hyphens, parentheses, and spaces. All suggestions will be considered; however, not all assignment suggestions can be honored.

Information about Study Sections can be found here: \url{https://grants.nih.gov/grants/phs_assignment_information.htm\#StudySection}}

\subsection*{Rationale for assignment suggestions}

\instructions{Entry is limited to 1000 characters}



%%%%%%%%%%%%%%%%%%%%%%%%%%%%%%%%%%%%%%%%%%%%%%%%%%%%%%%%%%%%%%%%%%%%%%%%%%%%%%%
\clearpage
\section*{Project Summary/Abstract}

\instructions{ The Project Summary must contain a summary of the
    proposed activity suitable for dissemination to the public. It
    should be a self-contained description of the project and should
    contain a statement of objectives and methods to be employed. It
    should be informative to other persons working in the same or
    related fields and insofar as possible understandable to a
    scientifically or technically literate lay reader.  The Project
    Summary is meant to serve as a succinct and accurate description
    of the proposed work when separated from the application. State
    the application's broad, long-term objectives and specific aims,
    making reference to the health relatedness of the project (i.e.,
    relevance to the mission of the agency). Describe concisely the
    research design and methods for achieving the stated goals. This
    section should be informative to other persons working in the same
    or related fields and insofar as possible understandable to a
    scientifically or technically literate reader. Avoid describing
    past accomplishments and the use of the first person. Finally,
    please make every effort to be succinct. This section must be no
    longer than 30 lines of text.}


%%%%%%%%%%%%%%%%%%%%%%%%%%%%%%%%%%%%%%%%%%%%%%%%%%%%%%%%%%%%%%%%%%%%%%%%%%%%%%%
\clearpage
\section*{Project narrative}

\instructions{ Using no more than two or three sentences, describe the
    relevance of this research to public health. In this section, be
    succinct and use plain language that can be understood by a
    general, lay audience.}


%%%%%%%%%%%%%%%%%%%%%%%%%%%%%%%%%%%%%%%%%%%%%%%%%%%%%%%%%%%%%%%%%%%%%%%%%%%%%%%
\clearpage
\section*{Specific aims}

\instructions{State concisely the goals of the proposed research and
    summarize the expected outcome(s), including the impact that the
    results of the proposed research will exert on the research
    field(s) involved.  List succinctly the specific objectives of the
    research proposed, e.g., to test a stated hypothesis, create a
    novel design, solve a specific problem, challenge an existing
    paradigm or clinical practice, address a critical barrier to
    progress in the field, or develop new technology.}


%%%%%%%%%%%%%%%%%%%%%%%%%%%%%%%%%%%%%%%%%%%%%%%%%%%%%%%%%%%%%%%%%%%%%
\clearpage
\section*{Research strategy}

\instructions{ Organize the Research Strategy in the specified order
    and using the instructions provided below. Start each section with
    the appropriate section heading---Significance, Innovation,
    Approach. Cite published experimental details in the Research
    Strategy section and provide the full reference in the
    Bibliography and References cited section.}

%%%%%%%%%%%%%%%%%%%%%%%%%%%%%%%%%%%%%%%%%%%%%%%%%%%%%%%%%%%%%%%%%%%%%
\subsection*{Significance}

\instructions{ Instructions: Explain the importance of the problem or
    critical barrier to progress in the field that the proposed
    project addresses.  Explain how the proposed project will improve
    scientific knowledge, technical capability, and/or clinical
    practice in one or more broad fields.  Describe how the concepts,
    methods, technologies, treatments, services, or preventative
    interventions that drive this field will be changed if the
    proposed aims are achieved.}

\instructions{ Review criteria: Does the project address an important
    problem or a critical barrier to progress in the field? If the
    aims of the project are achieved, how will scientific knowledge,
    technical capability, and/or clinical practice be improved? How
    will successful completion of the aims change the concepts,
    methods, technologies, treatments, services, or preventative
    interventions that drive this field?}

%%%%%%%%%%%%%%%%%%%%%%%%%%%%%%%%%%%%%%%%%%%%%%%%%%%%%%%%%%%%%%%%%%%%%
\subsection*{Innovation}

\instructions{ Instructions: Explain how the application challenges and
    seeks to shift current research or clinical practice paradigms.
    Describe any novel theoretical concepts, approaches or
    methodologies, instrumentation or intervention(s) to be developed
    or used, and any advantage over existing methodologies,
    instrumentation or intervention(s).  Explain any refinements,
    improvements, or new applications of theoretical concepts,
    approaches or methodologies, instrumentation or interventions.}

\instructions{ Review criteria: Does the application challenge and seek
    to shift current research or clinical practice paradigms by
    utilizing novel theoretical concepts, approaches or methodologies,
    instrumentation, or interventions? Are the concepts, approaches or
    methodologies, instrumentation, or interventions novel to one
    field of research or novel in a broad sense? Is a refinement,
    improvement, or new application of theoretical concepts,
    approaches or methodologies, instrumentation, or interventions
    proposed? }


%%%%%%%%%%%%%%%%%%%%%%%%%%%%%%%%%%%%%%%%%%%%%%%%%%%%%%%%%%%%%%%%%%%%%
\subsection*{Approach}

\instructions{ Instructions: Describe the overall strategy,
    methodology, and analyses to be used to accomplish the specific
    aims of the project. Unless addressed separately in Item 5.5.15,
    include how the data will be collected, analyzed, and interpreted
    as well as any resource sharing plans as appropriate.  Discuss
    potential problems, alternative strategies, and benchmarks for
    success anticipated to achieve the aims.  If the project is in the
    early stages of development, describe any strategy to establish
    feasibility, and address the management of any high risk aspects
    of the proposed work.}

\instructions {For new applications, include information on
  Preliminary Studies as part of the Approach section. Discuss the
  PD/PI's preliminary studies, data, and/or experience pertinent to
  this application. Except for Exploratory/Development Grants
  (R21/R33), Small Research Grants (R03), Academic Research
  Enhancement Award (AREA) Grants (R15), and Phase I Small Business
  Research Grants (R41/R43), preliminary data can be an essential part
  of a research grant application and help to establish the likelihood
  of success of the proposed project.}

\instructions{ Review criteria: Are the overall strategy, methodology,
    and analyses well-reasoned and appropriate to accomplish the
    specific aims of the project? Are potential problems, alternative
    strategies, and benchmarks for success presented? If the project
    is in the early stages of development, will the strategy establish
    feasibility and will particularly risky aspects be managed?}

\subsubsection*{Aim 1: }

\myparagraph{Motivation}

\myparagraph{Methodology}

\myparagraph{Benchmarks for success}

\myparagraph{Alternative strategies}

\subsubsection*{Aim 2: }

\myparagraph{Motivation}

\myparagraph{Methodology}

\myparagraph{Benchmarks for success}

\myparagraph{Alternative strategies}

\subsubsection*{Aim 3: }

\myparagraph{Motivation}

\myparagraph{Methodology}

\myparagraph{Benchmarks for success}

\myparagraph{Alternative strategies}


%%%%%%%%%%%%%%%%%%%%%%%%%%%%%%%%%%%%%%%%%%%%%%%%%%%%%%%%%%%%%%%%%%%%%%%%%
\clearpage

\instructions{Provide a bibliography of any references cited in the
  Research Plan. Each reference must include names of all authors (in
  the same sequence in which they appear in the publication), the
  article and journal title, book title, volume number, page numbers,
  and year of publication. Include only bibliographic
  citations. Follow scholarly practices in providing citations for
  source materials relied upon in preparing any section of the
  application.  The references should be limited to relevant and
  current literature. While there is not a page limitation, it is
  important to be concise and to select only those literature
  references pertinent to the proposed research.}

\instructions{When citing articles that fall under the Public Access
  Policy, were authored or co- authored by the applicant and arose
  from NIH support, provide the NIH Manuscript Submission reference
  number (e.g., NIHMS97531) or the PubMed Central (PMC) reference
  number (e.g., PMCID234567) for each article. If the PMCID is not yet
  available because the Journal submits articles directly to PMC on
  behalf of their authors, indicate ``PMC Journal - In Process.''
  Citations that are not covered by the Public Access Policy, but are
  publicly available in a free, online format may include URLs or
  PMCID numbers along with the full reference (note that copies of
  these publications are not accepted as appendix material, see
  Section 5.7).}

% Rename the bibliography section.
\bibliographystyle{unsrt}
\renewcommand{\refname}{Bibliography \& References Cited}
\bibliography{refs}

%%%%%%%%%%%%%%%%%%%%%%%%%%%%%%%%%%%%%%%%%%%%%%%%%%%%%%%%%%%%%%%%%%%%%%%%%%%%%%
\clearpage
\section*{Resource sharing plan}

\instructions{Investigators seeking \$500,000 or more in direct costs
  (exclusive of consortium F\&A) in any budget period are expected to
  include a brief 1-paragraph description of how final research data
  will be shared, or explain why data-sharing is not possible (for
  example human subject concerns, the Small Business Innovation
  Development Act provisions, etc.). Specific FOAs may require that
  all applications include this information regardless of the dollar
  level. Applicants are encouraged to read the FOA carefully and
  discuss their data-sharing plan with their program contact at the
  time they negotiate an agreement with the Institute/Center (IC)
  staff to accept assignment of their application.}

\subsection*{Data sharing plan}


\subsection*{Software dissemination plan}

\instructions{A software dissemination plan, with appropriate
  timelines, must be included in the application. There is no
  prescribed single license for software produced through grants
  responding to this announcement.  However, NIH does have goals for
  software dissemination, and reviewers will be instructed to evaluate
  the dissemination plan relative to these goals:
\begin{enumerate}
\item The software should be freely available to biomedical
  researchers and educators in the non-profit sector, such as
  institutions of education, research institutions, and government
  laboratories.
\item The terms of software availability should permit the
  dissemination and commercialization of enhanced or customized
  versions of the software, or incorporation of the software or pieces
  of it into other software packages.
\item To preserve utility to the community, the software should be
  transferable such that another individual or team can continue
  development in the event that the original investigators are
  unwilling or unable to do so.
\item The terms of software availability should include the ability of
  researchers to modify the source code and to share modifications
  with other colleagues.  An applicant should take responsibility for
  creating the original and subsequent ``official'' versions of a
  piece of software.
\item To further enhance the potential impact of their software,
  applicants may consider proposing a plan to manage and disseminate
  the improvements or customizations of their tools and resources by
  others.  This proposal may include a plan to incorporate the
  enhancements into the ``official'' core software, may involve the
  creation of an infrastructure for plug-ins, or may describe some
  other solution.
\end{enumerate}
The adequacy of the software sharing plans will be considered by
Program staff when making recommendations about funding applications.
In making such considerations, prior to funding, program staff may
negotiate modifications of software sharing plans with the Principal
Investigator before recommending funding of an application. Any
software dissemination plans represent a commitment by the institution
(and its subcontractors as applicable) to support and abide by the
plan.  The final version of any accepted software sharing plans will
become a condition of the award of the grant. The effectiveness of
software sharing may be evaluated as part of the administrative review
of each Non-Competing Grant Progress Report (PHS 2590).}


%%%%%%%%%%%%%%%%%%%%%%%%%%%%%%%%%%%%%%%%%%%%%%%%%%%%%%%%%%%%%%%%%%%%%%%%%%%%%%
\clearpage
\section*{Equipment}


%%%%%%%%%%%%%%%%%%%%%%%%%%%%%%%%%%%%%%%%%%%%%%%%%%%%%%%%%%%%%%%%%%%%%%%%%%%%%%
\clearpage
\section*{Budget justification}

\paragraph{Personnel}

\paragraph{Equipment}

\paragraph{Travel}

\paragraph{Other costs}

\paragraph{Online storage and backup}

\paragraph{Computing services}
 
\paragraph{Benefit and Escalation Rates}
\begin{itemize}
\item \% for faculty
\item \% for professional staff
\item \% for postdoctoral associates
\item \% for graduate students
\end{itemize}

\paragraph{F \& A Rate}
The current F \& A rate of XX.X\% is a negotiated agreement between DHHS and the University of Washington in a letter dated XXX.

\paragraph{Tuition}

%%%%%%%%%%%%%%%%%%%%%%%%%%%%%%%%%%%%%%%%%%%%%%%%%%%%%%%%%%%%%%%%%%%%%%%%%%%%%%
\clearpage
\section*{Resources}

\instructions{ Instructions: This information is used to assess the
    capability of the organizational resources available to perform
    the effort proposed.  Identify the facilities to be used
    (laboratory, clinical, animal, computer, office, other). If
    appropriate, indicate their capacities, pertinent capabilities,
    relative proximity and extent of availability to the
    project. Describe only those resources that are directly
    applicable to the proposed work. Provide any information
    describing the Other Resources available to the project (e.g.,
    machine shop, electronic shop) and the extent to which they would
    be available to the project.  Describe how the scientific
    environment in which the research will be done contributes to the
    probability of success (e.g., institutional support, physical
    resources, and intellectual rapport). In describing the scientific
    environment in which the work will be done, discuss ways in which
    the proposed studies will benefit from unique features of the
    scientific environment or subject populations or will employ
    useful collaborative arrangements.}

\instructions{ Review criteria: Will the scientific environment in
    which the work will be done contribute to the probability of
    success? Are the institutional support, equipment and other
    physical resources available to the investigators adequate for the
    project proposed? Will the project benefit from unique features of
    the scientific environment, subject populations, or collaborative
    arrangements?}

\paragraph{Computer}

\paragraph{Office}

\paragraph{Other}


\end{document}
