\documentclass[12pt]{article}
\usepackage[sort&compress]{natbib}
\usepackage[yyyymmdd,hhmmss]{datetime}

% Use Helvetica font.
\usepackage{helvet}
\renewcommand{\familydefault}{\sfdefault}

% Use 1/2-inch margins.
\usepackage[margin=0.5in]{geometry}

% No page numbers.
\pagestyle{empty}

\newcommand{\instructions}[1]{}
% Before submitting, comment out this line.
%\renewcommand{\instructions}[1]{{\scriptsize \sc #1}}

% The title has an 81-character maximum.
\title{INSERT TITLE HERE}

\author{MY NAME\\
MY AFFILIATION}

\begin{document}

\maketitle

\begin{center}
An NIH R01 proposal, submitted in response to \\
{\em PAR-XX-XXX: INSERT TITLE OF PAR HERE}.
\end{center}

\vfill

{\scriptsize Compiled on \today\ at \currenttime.}

%%%%%%%%%%%%%%%%%%%%%%%%%%%%%%%%%%%%%%%%%%%%%%%%%%%%%%%%%%%%%%%%%%%%%%%%%%%%%%%
\clearpage
\section*{Project Summary/Abstract}

\instructions{ The Project Summary must contain a summary of the
    proposed activity suitable for dissemination to the public. It
    should be a self-contained description of the project and should
    contain a statement of objectives and methods to be employed. It
    should be informative to other persons working in the same or
    related fields and insofar as possible understandable to a
    scientifically or technically literate lay reader.  The Project
    Summary is meant to serve as a succinct and accurate description
    of the proposed work when separated from the application. State
    the application's broad, long-term objectives and specific aims,
    making reference to the health relatedness of the project (i.e.,
    relevance to the mission of the agency). Describe concisely the
    research design and methods for achieving the stated goals. This
    section should be informative to other persons working in the same
    or related fields and insofar as possible understandable to a
    scientifically or technically literate reader. Avoid describing
    past accomplishments and the use of the first person. Finally,
    please make every effort to be succinct. This section must be no
    longer than 30 lines of text.}


%%%%%%%%%%%%%%%%%%%%%%%%%%%%%%%%%%%%%%%%%%%%%%%%%%%%%%%%%%%%%%%%%%%%%%%%%%%%%%%
\clearpage
\section*{Project narrative}

\instructions{ Using no more than two or three sentences, describe the
    relevance of this research to public health. In this section, be
    succinct and use plain language that can be understood by a
    general, lay audience.}

2%%%%%%%%%%%%%%%%%%%%%%%%%%%%%%%%%%%%%%%%%%%%%%%%%%%%%%%%%%%%%%%%%%%%%%%%%%%%%%%
\clearpage
\section*{Specific aims}

\instructions{State concisely the goals of the proposed research and
    summarize the expected outcome(s), including the impact that the
    results of the proposed research will exert on the research
    field(s) involved.  List succinctly the specific objectives of the
    research proposed, e.g., to test a stated hypothesis, create a
    novel design, solve a specific problem, challenge an existing
    paradigm or clinical practice, address a critical barrier to
    progress in the field, or develop new technology.}


%%%%%%%%%%%%%%%%%%%%%%%%%%%%%%%%%%%%%%%%%%%%%%%%%%%%%%%%%%%%%%%%%%%%%
\clearpage
\section*{Research strategy}

\instructions{ Organize the Research Strategy in the specified order
    and using the instructions provided below. Start each section with
    the appropriate section heading---Significance, Innovation,
    Approach. Cite published experimental details in the Research
    Strategy section and provide the full reference in the
    Bibliography and References cited section.}

%%%%%%%%%%%%%%%%%%%%%%%%%%%%%%%%%%%%%%%%%%%%%%%%%%%%%%%%%%%%%%%%%%%%%
\subsection*{Significance}

\instructions{ Instructions: Explain the importance of the problem or
    critical barrier to progress in the field that the proposed
    project addresses.  Explain how the proposed project will improve
    scientific knowledge, technical capability, and/or clinical
    practice in one or more broad fields.  Describe how the concepts,
    methods, technologies, treatments, services, or preventative
    interventions that drive this field will be changed if the
    proposed aims are achieved.}

\instructions{ Review criteria: Does the project address an important
    problem or a critical barrier to progress in the field? If the
    aims of the project are achieved, how will scientific knowledge,
    technical capability, and/or clinical practice be improved? How
    will successful completion of the aims change the concepts,
    methods, technologies, treatments, services, or preventative
    interventions that drive this field?}

%%%%%%%%%%%%%%%%%%%%%%%%%%%%%%%%%%%%%%%%%%%%%%%%%%%%%%%%%%%%%%%%%%%%%
\subsection*{Innovation}

\instructions{ Instructions: Explain how the application challenges and
    seeks to shift current research or clinical practice paradigms.
    Describe any novel theoretical concepts, approaches or
    methodologies, instrumentation or intervention(s) to be developed
    or used, and any advantage over existing methodologies,
    instrumentation or intervention(s).  Explain any refinements,
    improvements, or new applications of theoretical concepts,
    approaches or methodologies, instrumentation or interventions.}

\instructions{ Review criteria: Does the application challenge and seek
    to shift current research or clinical practice paradigms by
    utilizing novel theoretical concepts, approaches or methodologies,
    instrumentation, or interventions? Are the concepts, approaches or
    methodologies, instrumentation, or interventions novel to one
    field of research or novel in a broad sense? Is a refinement,
    improvement, or new application of theoretical concepts,
    approaches or methodologies, instrumentation, or interventions
    proposed? }

%%%%%%%%%%%%%%%%%%%%%%%%%%%%%%%%%%%%%%%%%%%%%%%%%%%%%%%%%%%%%%%%%%%%%
\subsection*{Approach}

\instructions{ Instructions: Describe the overall strategy,
    methodology, and analyses to be used to accomplish the specific
    aims of the project. Unless addressed separately in Item 5.5.15,
    include how the data will be collected, analyzed, and interpreted
    as well as any resource sharing plans as appropriate.  Discuss
    potential problems, alternative strategies, and benchmarks for
    success anticipated to achieve the aims.  If the project is in the
    early stages of development, describe any strategy to establish
    feasibility, and address the management of any high risk aspects
    of the proposed work.}

\instructions {For new applications, include information on
  Preliminary Studies as part of the Approach section. Discuss the
  PD/PI'ェs preliminary studies, data, and/or experience pertinent to
  this application. Except for Exploratory/Development Grants
  (R21/R33), Small Research Grants (R03), Academic Research
  Enhancement Award (AREA) Grants (R15), and Phase I Small Business
  Research Grants (R41/R43), preliminary data can be an essential part
  of a research grant application and help to establish the likelihood
  of success of the proposed project.}

\instructions{ Review criteria: Are the overall strategy, methodology,
    and analyses well-reasoned and appropriate to accomplish the
    specific aims of the project? Are potential problems, alternative
    strategies, and benchmarks for success presented? If the project
    is in the early stages of development, will the strategy establish
    feasibility and will particularly risky aspects be managed?}

\subsubsection*{Preliminary studies}

\subsubsection*{Specific Aim 1:}

\subsubsection*{Specific Aim 2:}

%%%%%%%%%%%%%%%%%%%%%%%%%%%%%%%%%%%%%%%%%%%%%%%%%%%%%%%%%%%%%%%%%%%%%%%%%
\clearpage

\instructions{Provide a bibliography of any references cited in the
  Research Plan. Each reference must include names of all authors (in
  the same sequence in which they appear in the publication), the
  article and journal title, book title, volume number, page numbers,
  and year of publication. Include only bibliographic
  citations. Follow scholarly practices in providing citations for
  source materials relied upon in preparing any section of the
  application.  The references should be limited to relevant and
  current literature. While there is not a page limitation, it is
  important to be concise and to select only those literature
  references pertinent to the proposed research.}

\instructions{When citing articles that fall under the Public Access
  Policy, were authored or co- authored by the applicant and arose
  from NIH support, provide the NIH Manuscript Submission reference
  number (e.g., NIHMS97531) or the PubMed Central (PMC) reference
  number (e.g., PMCID234567) for each article. If the PMCID is not yet
  available because the Journal submits articles directly to PMC on
  behalf of their authors, indicate ``PMC Journal - In Process.''
  Citations that are not covered by the Public Access Policy, but are
  publicly available in a free, online format may include URLs or
  PMCID numbers along with the full reference (note that copies of
  these publications are not accepted as appendix material, see
  Section 5.7).}

% Rename the bibliography section.
\bibliographystyle{unsrt}
\renewcommand{\refname}{Bibliography \& References Cited}
\bibliography{refs}

%%%%%%%%%%%%%%%%%%%%%%%%%%%%%%%%%%%%%%%%%%%%%%%%%%%%%%%%%%%%%%%%%%%%%%%%%
\clearpage
\section*{Resource sharing plan}

\subsection*{Data sharing plan}

%%%%%%%%%%%%%%%%%%%%%%%%%%%%%%%%%%%%%%%%%%%%%%%%%%%%%%%%%%%%%%%%%%%%%%%%%
\clearpage
\section*{Budget justification}

\paragraph{Personnel}

\begin{itemize}

\item 
\end{itemize}

%%%%%%%%%%%%%%%%%%%%%%%%%%%%%%%%%%%%%%%%%%%%%%%%%%%%%%%%%%%%%%%%%%%%%%%%%
\clearpage
\section*{Resources}

\instructions{ Instructions: This information is used to assess the
    capability of the organizational resources available to perform
    the effort proposed.  Identify the facilities to be used
    (laboratory, clinical, animal, computer, office, other). If
    appropriate, indicate their capacities, pertinent capabilities,
    relative proximity and extent of availability to the
    project. Describe only those resources that are directly
    applicable to the proposed work. Provide any information
    describing the Other Resources available to the project (e.g.,
    machine shop, electronic shop) and the extent to which they would
    be available to the project.  Describe how the scientific
    environment in which the research will be done contributes to the
    probability of success (e.g., institutional support, physical
    resources, and intellectual rapport). In describing the scientific
    environment in which the work will be done, discuss ways in which
    the proposed studies will benefit from unique features of the
    scientific environment or subject populations or will employ
    useful collaborative arrangements.}

\instructions{ Review criteria: Will the scientific environment in
    which the work will be done contribute to the probability of
    success? Are the institutional support, equipment and other
    physical resources available to the investigators adequate for the
    project proposed? Will the project benefit from unique features of
    the scientific environment, subject populations, or collaborative
    arrangements?}

\paragraph{Office}

\paragraph{Computer}

\paragraph{Other}

\paragraph{Major Equipment}

\end{document}
