\documentclass[11pt]{article}
\usepackage[sort&compress]{natbib}
\usepackage[yyyymmdd,hhmmss]{datetime}
\usepackage[hidelinks]{hyperref}

\usepackage{booktabs}
\usepackage{multirow}
\usepackage{graphicx}
\usepackage{caption}
\usepackage{subcaption}
\usepackage{enumitem}
\setlist{nolistsep}
\usepackage[normalem]{ulem}
\usepackage[compact]{titlesec}       
\titlespacing{\section}{-0.1pt}{-0.1pt}{-0.1pt}
\titlespacing{\subsection}{-0.1pt}{-0.1pt}{-0.1pt}
\titlespacing{\subsubsection}{-0.1pt}{-0.1pt}{-0.1pt}
\titlespacing{\paragraph}{-0.1pt}{-0.1pt}{-0.1pt}

\newcounter{enumparanum}

\newcommand{\numberedpar}{%
    \refstepcounter{enumparanum}%
    \makebox[0pt][r]{\theenumparanum\hspace{1em}}%
}

% Timeline
\usepackage{pgfgantt}

%proper arabic enumerate
\renewcommand{\labelenumi}{\textbf{\Alph{enumi}.}}
\renewcommand{\labelenumii}{\textbf{\Alph{enumi}\arabic{enumii})}}
\renewcommand{\labelenumiii}{\textbf{\Alph{enumi}\arabic{enumii}.\arabic{enumiii})}}
\renewcommand{\labelenumiv}{\textbf{\Alph{enumi}\arabic{enumii}.\arabic{enumiii}.\arabic{enumiv})}}

% Use Helvetica font.
\usepackage{helvet}
\renewcommand{\familydefault}{\sfdefault}

% Use 1/2-inch margins.
\usepackage[margin=0.5in]{geometry}

% No page numbers.
%%\pagestyle{empty}
%%\usepackage{nopageno}

% Allow wrapping of text around figures.
\usepackage{wrapfig}
\usepackage{url}

% A space-saving version of \paragraph.
%\newcommand{\myparagraph}[1]{\vspace{0.5ex}{\bf #1} ~} 
\newcommand{\myparagraph}[1]{\vspace{0.5ex}\noindent{\bf #1} ~}

% Make captions have small font.
\usepackage[font=scriptsize]{caption}

\newcommand{\instructions}[1]{}

% Before submitting, comment out this line.
\renewcommand{\instructions}[1]{{\scriptsize \sc #1}}

\usepackage{xcolor}
\newcommand{\fixme}[1]{}
\renewcommand{\fixme}[1]{{\color{red} {#1}}\color{black}}

% Rotate 45 degrees.
\usepackage{rotating}
\newcommand{\rdg}[1]{\begin{rotate}{45} #1 \end{rotate}}

% Display a checkmark.
\usepackage{amssymb}
\newcommand{\vv}{\checkmark}

% Compact titles.
\usepackage[compact]{titlesec}

% Compacted lists.  Use itemize* or enumerate* for compact version.
\usepackage{mdwlist}

% Reduce space after figures.
\setlength{\belowcaptionskip}{-10pt}

\usepackage{authblk} % Better formatting of author list.

% The title has an 81-character maximum.
\title{Title Here.}


\author[1]{Author}

\affil[1]{Affiliation}

\date{}

\begin{document}

\maketitle

\begin{center}
An NIH R\#\# proposal, submitted in response to \\
{\em \href{web link here}{PA-\#\#-\#\#\#}: NIH Research Project Grant}.
\end{center}

\vfill

{\scriptsize Compiled on \today\ at \currenttime.}
%%%%%%%%%%%%%%%%%%%%%%%%%%%%%%%%%%%%%%%%%%%%%%%%%%%%%%%%%%%%%%%%%%%%%%%%%%%%%
\maketitle

%%%%%%%%%%%%%%%%%%%%%%%%%%%%%%%%%%%%%%%%%%%%%%%%%%%%%%%%%%%%%%%%%%%%%%%%%%%%%%%
\clearpage
\section*{PROJECT SUMMARY/ABSTRACT}

Paragraph 1

Paragraph 2

Paragraph 3

%%%%%%%%%%%%%%%%%%%%%%%%%%%%%%%%%%%%%%%%%%%%%%%%%%%%%%%%%%%%%%%%%%%%%%%%%%%%%%%
\clearpage
\section*{PROJECT NARRATIVE}
Paragraph 

%%%%%%%%%%%%%%%%%%%%%%%%%%%%%%%%%%%%%%%%%%%%%%%%%%%%%%%%%%%%%%%%%%%%%%%%%%%%%%%
%%%%%%%%%%%%%%%%%%%%%%%%%%%%%%%%%%%%%%%%%%%%%%%%%%%%%%%%%%%%%%%%%%%%%%%%%%%%%%%
\clearpage
\section*{SPECIFIC AIMS}
%%%%%%%%%%%%%%%%%%%%%%%%%%%%%%%%%%%%%%%%%%%%%%%%%%%%%%%%%%%%%%%%%%%%%%%%%%%%%%%
%%%%%%%%%%%%%%%%%%%%%%%%%%%%%%%%%%%%%%%%%%%%%%%%%%%%%%%%%%%%%%%%%%%%%%%%%%%%%%%
Paragraph 1

Paragraph 2

Paragraph 3

Paragraph 4

\myparagraph{Aim 1. } 
Paragraph

\myparagraph{Aim 2. }
Paragraph

\myparagraph{Aim 3. } 
Paragraph

%%%%%%%%%%%%%%%%%%%%%%%%%%%%%%%%%%%%%%%%%%%%%%%%%%%%%%%%%%%%%%%%%%%%%%%%%%%%%%%
%%%%%%%%%%%%%%%%%%%%%%%%%%%%%%%%%%%%%%%%%%%%%%%%%%%%%%%%%%%%%%%%%%%%%%%%%%%%%%%
\clearpage
\section*{RESEARCH STRATEGY} 
%%%%%%%%%%%%%%%%%%%%%%%%%%%%%%%%%%%%%%%%%%%%%%%%%%%%%%%%%%%%%%%%%%%%%%%%%%%%%%%
%%%%%%%%%%%%%%%%%%%%%%%%%%%%%%%%%%%%%%%%%%%%%%%%%%%%%%%%%%%%%%%%%%%%%%%%%%%%%%%

\begin{enumerate}[wide, labelindent=0pt]

%%%%%%%%%%%%%%%%%%%%%%%%%%%%%%%%%%%%%%%%%%%%%%%%%%%%%%%%%%%%%%%%%%%%%%%%%%%%%%%
\item \subsection*{SIGNIFICANCE} 
%%%%%%%%%%%%%%%%%%%%%%%%%%%%%%%%%%%%%%%%%%%%%%%%%%%%%%%%%%%%%%%%%%%%%%%%%%%%%%%
Paragraph 1

Paragraph 2

\begin{figure}[h]
    \vspace{-2mm}
    \begin{subfigure}[b]{0.45\textwidth}
    \includegraphics[height=4cm]{images/Placeholder1.png}
        \caption{Placeholder 1 caption.}
    \label{fig:placeholder_1_double}
    \end{subfigure}
    \hspace{0.06\textwidth}
    \begin{subfigure}[t]{0.45\textwidth}
        \includegraphics[height=4cm]{images/Placeholder2.png}
        \caption{Placeholder 2 caption.}
        \label{fig:placeholder_2_double}
    \end{subfigure}
    \caption{Figure caption}
    \label{fig:placeholder_1_2_double}
\end{figure}


Paragraph 3
The proposed experiments are significant because:

    %\begin{enumerate}[wide, labelindent=0pt, itemsep=1.5em, leftmargin=0pt] 
    \begin{enumerate}[wide, labelindent=0pt, leftmargin=0pt]
        \item \textit{\uline{First point underlined.}} 
        Paragraph
        \item \textit{\uline{Second point underlined.}} 
        Paragraph\\
    \end{enumerate} 

%%%%%%%%%%%%%%%%%%%%%%%%%%%%%%%%%%%%%%%%%%%%%%%%%%%%%%%%%%%%%%%%%%%%%%%%%%%%%%%
\item \subsection*{INNOVATION} 
%%%%%%%%%%%%%%%%%%%%%%%%%%%%%%%%%%%%%%%%%%%%%%%%%%%%%%%%%%%%%%%%%%%%%%%%%%%%%%%
    
    %\begin{enumerate}[wide, labelindent=0pt, itemsep=1.5em, leftmargin=0pt] 
    \begin{enumerate}[wide, labelindent=0pt, leftmargin=0pt] 
        \item \textit{\uline{First point underlined.}} 
        Paragraph
        \item \textit{\uline{Second point underlined.}} 
        Paragraph
        \item \textit{\uline{Third point underlined.}} 
        Paragraph\\
    \end{enumerate} 
    
%%%%%%%%%%%%%%%%%%%%%%%%%%%%%%%%%%%%%%%%%%%%%%%%%%%%%%%%%%%%%%%%%%%%%%%%%%%%%%%
\item \subsection*{APPROACH} 
%%%%%%%%%%%%%%%%%%%%%%%%%%%%%%%%%%%%%%%%%%%%%%%%%%%%%%%%%%%%%%%%%%%%%%%%%%%%%%%

Paragraph\\

    \begin{enumerate}[wide, labelindent=0pt]

%%%%%%%%%%%%%%%%%%%%%%%%%%%%%%%%%%%%%%%%%%%%%%%%%%%%%%%
    \item \subsubsection*{INTRODUCTION}
%%%%%%%%%%%%%%%%%%%%%%%%%%%%%%%%%%%%%%%%%%%%%%%%%%%%%%%
        Paragraph

        \underline{Topic 1. }
        Paragraph

        \underline{Topic 2. }
        Paragraph 1

        Paragraph 2

        \underline{Topic 3. } 
        Paragraph

%%%%%%%%%%%%%%%%%%%%%%%%%%%%%%%%%%%%%%%%%%%%%%%%%%%%%%%
    \item \subsubsection*{PRELIMINARY STUDIES}
%%%%%%%%%%%%%%%%%%%%%%%%%%%%%%%%%%%%%%%%%%%%%%%%%%%%%%%

        \underline{Results 1. } 
        Paragraph

        \underline{Results 2. }
        Paragraph

%%%%%%%%%%%%%%%%%%%%%%%%%%%%%%%%%%%%%%%%%%%%%%%%%%%%%%%
    \item \subsubsection*{RESEARCH PLAN}
%%%%%%%%%%%%%%%%%%%%%%%%%%%%%%%%%%%%%%%%%%%%%%%%%%%%%%%
%%%%%%%%%%%%%%%%
%%% AIM 1 %%%%%%
%%%%%%%%%%%%%%%% 
        
        \myparagraph{Aim 1. }
        Citation example. Lorem ipsum dolor sit amet, consectetur adipiscing elit. Sed do eiusmod tempor incididunt ut labore et dolore magna aliqua. Ut enim ad minim veniam, quis nostrud exercitation ullamco laboris nisi ut aliquip ex ea commodo consequat. Duis aute irure dolor in reprehenderit in voluptate velit esse cillum dolore eu fugiat nulla pariatur. Excepteur sint occaecat cupidatat non proident, sunt in culpa qui officia deserunt mollit anim id est laborum \cite{Doe2024}. 

        \myparagraph{Aim 1.1 } 
        \underline{Rational:} 
        Paragraph 1

        Paragraph 2
        
        \begin{figure}[h]
            \centering 
            \includegraphics[width=0.9\textwidth]{images/Placeholder1.png}
            \caption{Placeholder 1 single caption.}
            \label{fig:placeholder_1_single}
        \end{figure}

        \underline{Experimental Design.} 
        Experiments: 
        (i) experiment 1
        (ii) experiment 2

        \underline{Expected Outcomes.} 
        Paragraph

        \underline{Potential problems and Alternatives.} 
        Paragraph
        
        \myparagraph{Aim 1.2 } 
        \underline{Rational:} 
        Paragraph
        A \underline{hypothesis} sentence. 
        
        \underline{Experimental Design.}
        Experiments: 
        (i) experiment 1
        (ii) experiment 2

        \underline{Expected Outcomes.} 
        Paragraph
        
        \underline{Potential problems and Alternatives.} 
        Paragraph 
        
%%%%%%%%%%%%%%%%
%%% AIM 2 %%%%%%
%%%%%%%%%%%%%%%%
        
        \myparagraph{Aim 2. } 
        Paragraph

        Paragraph

        \myparagraph{Aim 2.1 }
        \underline{Rational:} 
        Paragraph

        \underline{Experimental Design.} 
        Experiments: 
        (i) experiment 1
        (ii) experiment 2

        \underline{Expected Outcomes.} 
        Paragraph

        \underline{Potential problems and Alternatives.} 
        Paragraph

        Paragraph
        
        \myparagraph{Aim 2.2 }
        \underline{Rational:} 
        Paragraph

        \underline{Experimental Design.} 
        Experiments: 
        (i) experiment 1
        (ii) experiment 2

        \underline{Expected Outcomes.} 
        Paragraph

        Paragraph

        \underline{Potential problems and Alternatives.} 
        Paragraph

        Paragraph

%%%%%%%%%%%%%%%%
%%% AIM 3 %%%%%%
%%%%%%%%%%%%%%%%
        
        \myparagraph{Aim 3. }
        Paragraph

        Paragraph

        Paragraph

        \myparagraph{Aim 3.1 }
        \underline{Rationale:} 
        Paragraph

        \underline{Experimental Design.} 
        Experiments: 
        (i) experiment 1
        (ii) experiment 2

        \underline{Expected Outcomes.} 
        Paragraph

        Paragraph

        \underline{Potential problems and Alternatives.} 
        Paragraph

        Paragraph

        \myparagraph{Aim 3.2 }
        \underline{Rational:} 
        Paragraph
        A \underline{hypothesis} sentence.
        
        \underline{Experimental Design.} 
        Experiments: 
        (i) experiment 1
        (ii) experiment 2
        
        \underline{Expected Outcomes.}
        Paragraph

        \underline{Potential problems and Alternatives.} 
        Paragraph
        
        
%%%%%%%%%%%%%%%%%%%%%%%%%%%%%%%%%%%%%%%%%%%%%%%%%%%%%%%
    \item \subsubsection*{OVERALL SUMMARY AND CONCLUSIONS}
%%%%%%%%%%%%%%%%%%%%%%%%%%%%%%%%%%%%%%%%%%%%%%%%%%%%%%%
        Lorem ipsum dolor sit amet, consectetur adipiscing elit. Sed do eiusmod tempor incididunt ut labore et dolore magna aliqua. Ut enim ad minim veniam, quis nostrud exercitation ullamco laboris nisi ut aliquip ex ea commodo consequat. Duis aute irure dolor in reprehenderit in voluptate velit esse cillum dolore eu fugiat nulla pariatur. Excepteur sint occaecat cupidatat non proident, sunt in culpa qui officia deserunt mollit anim id est laborum.

        % Example figures
        % Inline with text
        \noindent
        \begin{minipage}[c]{0.75\textwidth}\setlength{\parindent}{1.5em}
        Lorem ipsum dolor sit amet, consectetur adipiscing elit.
        Vestibulum ac urna ac odio malesuada tincidunt in id risus.
        Morbi eu lacus eget ligula sodales luctus ut at orci.
        Phasellus vitae nulla eu sem volutpat accumsan et eget purus. Curabitur nec ante at eros tincidunt mollis, e.g., interdum at pharetra lectus.
        Recent findings highlight that varius ultrices augue efficitur tristique velit, vestibulum consequat massa porttitor potenti.
        Lorem ipsum dolor sit amet, consectetur adipiscing elit.
        Vestibulum ac urna ac odio malesuada tincidunt in id risus.
        Morbi eu lacus eget ligula sodales luctus ut at orci \ref{fig:placeholder_1_inline}.
        Phasellus vitae nulla eu sem volutpat accumsan et eget purus. Curabitur nec ante at eros tincidunt mollis, e.g., interdum at pharetra lectus.
        Recent findings highlight that varius ultrices augue efficitur tristique velit, vestibulum consequat massa porttitor potenti.
        \end{minipage}
        %\hspace{0.02cm}
        % adjust textwidth, vspace, and scale as needed
        \begin{minipage}[c]{0.18\textwidth}
            \vspace{-24pt}
            \captionsetup{skip=0pt,labelfont=bf}
            \includegraphics[scale=0.14]{images/Placeholder1.png}
            \captionof{figure}{Inline placeholder 1 caption.}
            \label{fig:placeholder_1_inline}
        \end{minipage}\\
        
        % Three figures across page
        \begin{figure}[h]
            \begin{subfigure}[b]{0.3\textwidth}
            \includegraphics[height=4.5cm]{images/Placeholder1.png}
                \caption{Placeholder 1 caption triple.}
            \label{fig:placeholder_1_triple}
            \end{subfigure}
            \hspace{0.01\textwidth}
            \begin{subfigure}[b]{0.3\textwidth}
                \includegraphics[height=4.5cm]{images/Placeholder2.png}
                \caption{Placeholder 2 caption triple.}
                \label{fig:placeholder_2_triple}
            \end{subfigure}
            \hspace{0.01\textwidth}
            \begin{subfigure}[b]{0.3\textwidth}
                \includegraphics[height=4.5cm]{images/Placeholder3.png}
                \caption{Placeholder 3 caption triple.}
                \label{fig:placeholder_3_triple}
            \end{subfigure}
            \caption{Triple figures caption.}
            \label{fig:3_figures_row}
        \end{figure}
        

    \end{enumerate}
\end{enumerate}

%\hfill \break
%\noindent
\begin{figure}[h]
\centering
\tikz{% <<< TRICK: put ganttchart as node-content, draw borders of node
 \node [draw] {% next follows the content
     \begin{ganttchart}[%Specs
        [
        canvas/.append style={fill=none, draw=black!5, line width=0.75pt},
        hgrid style/.style={*1{draw=black!5, line width=0.75pt}},
        vgrid={*1{draw=black!5, line width=0.75pt}},
        x unit=15.5pt,
        y unit title=0.25cm,
        y unit chart=0.35cm,
        title/.style={draw=none, fill=none},
        title label font=\footnotesize,
        title label node/.append style={below=-10pt},
        title height=0.2,
        include title in canvas=false,
        bar label font=\mdseries\tiny\color{black!70},
        bar label node/.append style={left=1cm},
        bar/.append style={draw=none, fill=black!63},
        bar height=0.6,
        group left shift=0,
        group right shift=0,
        group height=0.5,
        group peaks tip position=0,
        group label node/.append style={left=.6cm},
        group label font=\bfseries\footnotesize,
        milestone label font=\mdseries\tiny\color{black!70},
        milestone label node/.append style={left=1cm},
        milestone/.append style={draw=none, fill=black!63},
        today label font=\tiny,
        vrule label font=\tiny,
        inline]{1}{25}
        % \gantttitle{\textbf{Project Timeline}}{30}\\ 
        \gantttitle{\textbf{Year1}}{6}                        
        \gantttitle{\textbf{Year2}}{12}
        \gantttitle{\textbf{Year3}}{6} \\  
        \gantttitle[title label font=\tiny,title label node/.append style={below=-6pt}]{Q3}{3}           
        \gantttitle[title label font=\tiny,title label node/.append style={below=-6pt}]{Q4}{3}
        \gantttitle[title label font=\tiny,title label node/.append style={below=-6pt}]{Q1}{3}
        \gantttitle[title label font=\tiny,title label node/.append style={below=-6pt}]{Q2}{3}
        \gantttitle[title label font=\tiny,title label node/.append style={below=-6pt}]{Q3}{3}
        \gantttitle[title label font=\tiny,title label node/.append style={below=-6pt}]{Q4}{3}
        \gantttitle[title label font=\tiny,title label node/.append style={below=-6pt}]{Q1}{3}
        \gantttitle[title label font=\tiny,title label node/.append style={below=-6pt}]{Q2}{3} \\

        \ganttmilestone[inline=false,milestone/.append style={fill=black}]{Manuscripts}{1}
        \ganttmilestone[inline=false,milestone/.append style={fill=black}]{}{10}
        \ganttmilestone[inline=false,milestone/.append style={fill=black}]{}{16}
        \ganttmilestone[inline=false,milestone/.append style={fill=black}]{}{17} 
        \ganttmilestone[inline=false,milestone/.append style={fill=black}]{}{23} \\
        % AIM 1
        \ganttgroup[inline=false,group/.append style={fill=cyan}] {\color{cyan}\underline{Aim 1: Here}}{8}{20} \\ 
        \ganttbar[inline=false,bar/.append style={fill=cyan!30}]{Step 1}{2}{17} \\
        \ganttbar[inline=false,bar/.append style={fill=cyan!30}]{Step 2}{9}{10} 
        \ganttbar[inline=true,bar/.append style={fill=cyan!30},bar label node/.append style={left=1cm}]{}{15}{16} 
        \ganttbar[inline=true,bar/.append style={fill=cyan!30},bar label node/.append style={left=1cm}]{}{22}{23} \\
        % Aim 2
        \ganttgroup[inline=false,group/.append style={fill=olive}]{\color{olive}\underline{Aim 2: Here}}{5}{18} \\ 
        \ganttbar[inline=false,bar/.append style={fill=olive!30}]{Step 1}{7}{15} \\
        \ganttbar[inline=false,bar/.append style={fill=olive!30}]{Step 2}{9}{16} \\
        \ganttbar[inline=false,bar/.append style={fill=olive!30}]{Step 3}{14}{17} 
        \ganttbar[inline=true,bar/.append style={fill=olive!30},bar label node/.append style={left=1cm}]{}{22}{23} \\

        \ganttgroup[inline=false,group/.append style={fill=violet},]{\color{violet}\underline{Aim 3 Here: }}{20}{24} \\ 
        \ganttbar[inline=false,bar/.append style={fill=violet!30}]{Step 1}{20}{24} \\
        \ganttbar[inline=false,bar/.append style={fill=violet!30}]{Step 2}{22}{24} \\

        \ganttvrule[vrule/.append style={cyan!20!black, thin}]{Year 3}{7}
        \ganttvrule[vrule/.append style={cyan!20!black, thin}]{Year 4}{19}
    \end{ganttchart}
    
 };
 % HEADER
}
    \caption{Project timeline description.}
    \label{fig:ganttchart} 
\end{figure}

%%%%%%%%%%%%%%%%%%%%%%%%%%%%%%%%%%%%%%%%%%%%%%%%%%%%%%%%%%%%%%%%%%%%%%%%%
\clearpage

\bibliographystyle{unsrt}
\renewcommand{\refname}{\raggedright Bibliography \& References Cited}
\bibliography{refs}


%%%%%%%%%%%%%%%%%%%%%%%%%%%%%%%%%%%%%%%%%%%%%%%%%%%%%%%%%%%%%%%%%%%%%%%%%%%%%%
\clearpage
\section*{Resource sharing plan}

\instructions{Investigators seeking \$500,000 or more in direct costs
  (exclusive of consortium F\&A) in any budget period are expected to
  include a brief 1-paragraph description of how final research data
  will be shared, or explain why data-sharing is not possible (for
  example human subject concerns, the Small Business Innovation
  Development Act provisions, etc.). Specific FOAs may require that
  all applications include this information regardless of the dollar
  level. Applicants are encouraged to read the FOA carefully and
  discuss their data-sharing plan with their program contact at the
  time they negotiate an agreement with the Institute/Center (IC)
  staff to accept assignment of their application.}

\subsection*{Data sharing plan}


\subsection*{Software dissemination plan}

\instructions{A software dissemination plan, with appropriate
  timelines, must be included in the application. There is no
  prescribed single license for software produced through grants
  responding to this announcement.  However, NIH does have goals for
  software dissemination, and reviewers will be instructed to evaluate
  the dissemination plan relative to these goals:
\begin{enumerate}
\item The software should be freely available to biomedical
  researchers and educators in the non-profit sector, such as
  institutions of education, research institutions, and government
  laboratories.
\item The terms of software availability should permit the
  dissemination and commercialization of enhanced or customized
  versions of the software, or incorporation of the software or pieces
  of it into other software packages.
\item To preserve utility to the community, the software should be
  transferable such that another individual or team can continue
  development in the event that the original investigators are
  unwilling or unable to do so.
\item The terms of software availability should include the ability of
  researchers to modify the source code and to share modifications
  with other colleagues.  An applicant should take responsibility for
  creating the original and subsequent ``official'' versions of a
  piece of software.
\item To further enhance the potential impact of their software,
  applicants may consider proposing a plan to manage and disseminate
  the improvements or customizations of their tools and resources by
  others.  This proposal may include a plan to incorporate the
  enhancements into the ``official'' core software, may involve the
  creation of an infrastructure for plug-ins, or may describe some
  other solution.
\end{enumerate}
The adequacy of the software sharing plans will be considered by
Program staff when making recommendations about funding applications.
In making such considerations, prior to funding, program staff may
negotiate modifications of software sharing plans with the Principal
Investigator before recommending funding of an application. Any
software dissemination plans represent a commitment by the institution
(and its subcontractors as applicable) to support and abide by the
plan.  The final version of any accepted software sharing plans will
become a condition of the award of the grant. The effectiveness of
software sharing may be evaluated as part of the administrative review
of each Non-Competing Grant Progress Report (PHS 2590).}


%%%%%%%%%%%%%%%%%%%%%%%%%%%%%%%%%%%%%%%%%%%%%%%%%%%%%%%%%%%%%%%%%%%%%%%%%%%%%%
\clearpage
\section*{Equipment}


%%%%%%%%%%%%%%%%%%%%%%%%%%%%%%%%%%%%%%%%%%%%%%%%%%%%%%%%%%%%%%%%%%%%%%%%%%%%%%
\clearpage
\section*{Budget justification}

\paragraph{Personnel}

\paragraph{Equipment}

\paragraph{Travel}

\paragraph{Other costs}

\paragraph{Online storage and backup}

\paragraph{Computing services}
 
\paragraph{Benefit and Escalation Rates}
\begin{itemize}
\item \% for faculty
\item \% for professional staff
\item \% for postdoctoral associates
\item \% for graduate students
\end{itemize}

\paragraph{F \& A Rate}
The current F \& A rate of XX.X\% is a negotiated agreement between DHHS and the University of Washington in a letter dated XXX.

\paragraph{Tuition}

%%%%%%%%%%%%%%%%%%%%%%%%%%%%%%%%%%%%%%%%%%%%%%%%%%%%%%%%%%%%%%%%%%%%%%%%%%%%%%
\clearpage
\section*{Resources}

\instructions{ Instructions: This information is used to assess the
    capability of the organizational resources available to perform
    the effort proposed.  Identify the facilities to be used
    (laboratory, clinical, animal, computer, office, other). If
    appropriate, indicate their capacities, pertinent capabilities,
    relative proximity and extent of availability to the
    project. Describe only those resources that are directly
    applicable to the proposed work. Provide any information
    describing the Other Resources available to the project (e.g.,
    machine shop, electronic shop) and the extent to which they would
    be available to the project.  Describe how the scientific
    environment in which the research will be done contributes to the
    probability of success (e.g., institutional support, physical
    resources, and intellectual rapport). In describing the scientific
    environment in which the work will be done, discuss ways in which
    the proposed studies will benefit from unique features of the
    scientific environment or subject populations or will employ
    useful collaborative arrangements.}

\instructions{ Review criteria: Will the scientific environment in
    which the work will be done contribute to the probability of
    success? Are the institutional support, equipment and other
    physical resources available to the investigators adequate for the
    project proposed? Will the project benefit from unique features of
    the scientific environment, subject populations, or collaborative
    arrangements?}

\paragraph{Computer}

\paragraph{Office}

\paragraph{Other}


\end{document}
